% Page style parameters
\addtolength{\oddsidemargin}{-.5in}
\addtolength{\evensidemargin}{-.5in}
\addtolength{\textwidth}{1in}
\addtolength{\topmargin}{-.5in}
\addtolength{\textheight}{1.0in}
\setlength{\headheight}{20pt}
\setlength{\headsep}{20pt}
\raggedbottom
\clubpenalty=8000 \widowpenalty=8000 % I hate orphans & widows

\makeatletter  %force abstract to appear on the same page as the title
	\def\titlepage{\@restonecolfalse\thispagestyle{empty}\c@page\z@}
	\def\endtitlepage{}
\makeatother

% Section numbering & TOC style parameters
\setcounter{secnumdepth}{4}
\setcounter{tocdepth}{3}

% Paragraph style parameters
\setlength{\parindent}{0pt}
\setlength{\parskip}{1.5ex}
\newskip  \normalparskip        
\normalparskip = 1.5ex

%\tracingall

%%
%% Basic macros
%%

	\makeatletter	% allow @-sign to be considered as a letter; for special definitions

%%% The following \cl macros are based on a version lifted from the CLIM 2 spec.
%% This allows hyphenation of CL symbols at the hyphens that exist in
%% the symbol name, but noplace else, e.g., \cl{standard-region-union}.
%% Hyphens are active characters, so we don't break Tags Search.
%% This version of \cl has additional smarts to allow TeX's special characters
%% (except braces and "\"), based on LaTeX's verbatim.
%% When you supply a default value for optional or keyword arguments,
%% use this idiom:  \key (filled \cl{t})

% The \doclspecials is a modified version of \dospecials from plain.tex
\def\doclspecials{\do\$\do\&\do\#\do\^\do\_\do\%\do\~}
% (not counting ascii null, tab, linefeed, formfeed, return, delete).
% For \cl we also removed the space, \, {, }, ^^A, and ^^K.
% Each symbol in the list is preceded by \do, which can be defined
% if you want to do something to every item in the list.

% The clhyphen definitions are necessary to support the automatic cl indexing macros.
% \index does strange things to its args, and the hyphen mustn't get expanded.
\def\cl@normalhyphen{-}
\def\cl@breakinghyphen{{\tt \char`\-}\penalty\exhyphenpenalty}
\def\cl@hyphen{\cl@normalhyphen}
\def\setclnormalhyphen{\let\cl@hyphen=\cl@normalhyphen}
\def\setclbreakinghyphen{\let\cl@hyphen=\cl@breakinghyphen}

{\catcode`\-=\active%
\gdef\setclspecials{\catcode``=\active\@noligs\frenchspacing\@vobeyspaces%
\setclbreakinghyphen\catcode`\-=\active\def-{\cl@hyphen}\def\\{\char`\\}%
\let\do=\@makeother\doclspecials}
\gdef\arg{\begingroup\setclspecials%
\def-{{\sl\char`\-}\penalty\exhyphenpenalty}%
\@xarg}}
\def\@xarg#1{{\sl#1}\endgroup}
\def\cl{\begingroup\setclspecials\@xcl}
\def\@xcl#1{{\tt#1}\endgroup}

%%% \begin{\clcode} ... \end{clcode} environment, based on LaTeX's verbatim.
\usepackage{code-os}
\def\clcode{\codeaux{}}
\let\endclcode=\endcodeaux

\def\ttdash{\mbox{{\tt -}}}	%for use in math mode

%\begingroup
%  \catcode`|=0 \catcode`[= 1 \catcode`]=2 \catcode`\{=12 \catcode`\}=12 \catcode`\\=12
%  |gdef|@xclcode#1\end{clcode}[#1|end[clcode]]
%|endgroup
%\def\clcode{\trivlist \item[]\if@minipage\else\vskip\parskip\fi
%  \leftskip\@totalleftmargin\rightskip\z@
%  \parindent\z@\parfillskip\@flushglue\parskip\z@
%  \@tempswafalse \def\par{\if@tempswa\hbox{}\fi\@tempswatrue\@@par}
%  \obeylines \tt \catcode``=\active \@noligs \frenchspacing\@vobeyspaces
%  \let\do=\@makeother \doclspecials \@xclcode}
%\let\endclcode=\endtrivlist

%%
%% Special indexing macros
%%

%\def\dowithonearg#1{\def\argone{#1}\do}
%\def\dowithtwoargs#1#2{\def\argone{#1}\def\argtwo{#2}\do}

%% inline references that are indexed

% singly indexed
% common lisp
\def\clinx{\begingroup\setclspecials\@xclinx}
	\def\@xclinx#1{\@xcl{#1}\@bsphack\setclnormalhyphen\index{{\tt#1}}\@esphack}

% ...d variants are doubly indexed
\def\cldinx{\begingroup\setclspecials\@xcldinx}
	\def\@xcldinx#1#2{\@xcl{#1}\ttdblinx{#1}{#2}}

	\def\ttdblinx#1#2{\@bsphack\begingroup\setclspecials\setclnormalhyphen%
	\index{{\tt#1}!#2}\index{#2!{\tt#1}}\endgroup\@esphack}

% macros, functions, ...
\def\inxmac{\begingroup\setclspecials\@xinxmac}
	\def\@xinxmac#1{\@xcldinx{#1}{Macro}}

\def\inxfun{\begingroup\setclspecials\@xinxfun}
	\def\@xinxfun#1{\@xcldinx{#1}{Function}}

\def\inxgeneric{\begingroup\setclspecials\@xinxgeneric}
	\def\@xinxgeneric#1{\@xcldinx{#1}{Generic Function}}

\def\inxmethod{\begingroup\setclspecials\@xinxmethod}
	\def\@xinxmethod#1{\@xcldinx{#1}{Primary Method}}

\def\inxbeforemethod{\begingroup\setclspecials\@xinxbeforemethod}
	\def\@xinxbeforemethod#1{\@xcldinx{#1}{:Before Method}}

\def\inxaftermethod{\begingroup\setclspecials\@xinxaftermethod}
	\def\@xinxaftermethod#1{\@xcldinx{#1}{:After Method}}

\def\inxaroundmethod{\begingroup\setclspecials\@xinxaroundmethod}
	\def\@xinxaroundmethod#1{\@xcldinx{#1}{:Around Method}}

\def\inxvar{\begingroup\setclspecials\@xinxvar}
	\def\@xinxvar#1{\@xcldinx{#1}{Variable}}

\def\inxconst{\begingroup\setclspecials\@xinxconst}
	\def\@xinxconst#1{\@xcldinx{#1}{Constant}}

\def\inxclass{\begingroup\setclspecials\@xinxclass}
	\def\@xinxclass#1{\@xcldinx{#1}{Class}}

\def\inxslot{\begingroup\setclspecials\@xinxslot}
	\def\@xinxslot#1{\@xcldinx{#1}{Slot}}

\def\inxinitarg{\begingroup\setclspecials\@xinxinitarg}
	\def\@xinxinitarg#1{\@xcldinx{#1}{Initarg}}

\def\inxtype{\begingroup\setclspecials\@xinxtype}
	\def\@xinxtype#1{\@xcldinx{#1}{Type}}

%%
%% Miscellaneous
%%


\long\def\issue#1#2{{\sl Issue: #2 ---~#1}}
\def\bold#1{{\bf #1}}  % italic.sty also defines \ital{} and \slant{}


% \comment<some-char>...text...<same-char> comments out TeX source
% analogous to \verb<some-char>...text...<same-char>
% sp. case: \comment{...text...} == \comment}...text...}
\def\comment{%   Stolen from slatex.sty
\begingroup \let\do\@makeother \dospecials \@comm}
\def\word@nd{\hskip-.35em\penalty\@M\null}
\begingroup\catcode`\<1\catcode`\>2
\catcode`\{12\catcode`\}12
\long\gdef\@comm#1<%
\if#1{\long\def\@tempa ##1}<\endgroup\word@nd>\else
	\long\def\@tempa ##1#1<\endgroup\word@nd>\fi
\@tempa>
\endgroup

%% Use \concept when you are introducing a term for the first time, and
%% want it in the index.  Use \term after that.
\def\concept#1{{\sf#1}\@bsphack\index{#1|ital}\@esphack}
\def\term#1{{\sf#1}}

	\makeatother	% reverses the effect of \makeatletter

%%
%% Definition types
%%

\def\removedepth{\ifdim \prevdepth>-1000pt \vskip -\prevdepth\fi}
\def\Vskip #1!{\endgraf\removedepth     
               \ifdim \lastskip<#1 \ifdim \lastskip>0pc
               \removelastskip\fi 
               \vskip#1\fi}

\def\outdent#1{\noindent\hbox to 0pc{\hskip -1.5pc #1\hss}\ignorespaces}
\def\brac#1{\rm[{\it #1\/}]}

%% Use these when you want to leave vertical whitespace above
\def\Dodoc#1#2{\Vskip \bigskipamount!%
	\outdent{$\Rightarrow$}\index{{\tt #1}!#2|ital}\index{#2!{\tt #1}|ital}%
	{\tt #1\hfill\hbox{\brac{\it #2\/}}%
		\Vskip .1pt!}}
\def\Dodocf#1#2#3{\Vskip \bigskipamount!%
	\outdent{$\Rightarrow$}\index{{\tt #1}!#3|ital}\index{#3!{\tt #1}|ital}%
	{\hangindent=1cm\hangafter=1%
		\tt #1\hfill\hbox{\brac{\it #3\/}}\linebreak%
		{\raggedright{\sl #2}\hfill\Vskip .1pt!}}}
%\def\Dodocv#1#2#3{\vbox spread -10pt {\Vskip \bigskipamount!%
\def\Dodocv#1#2#3{\vbox spread -2pt {\Vskip \bigskipamount!%
	\outdent{$\Rightarrow$}\index{{\tt #1}!#3|ital}\index{#3!{\tt #1}|ital}%
	{\hangindent=1cm\hangafter=1%
		\tt #1\hfill\hbox{\brac{\it #3\/}}\linebreak%
		{\raggedright{\tt #2}\hfill\Vskip .1pt!}}\vss}}

%% Use these when you want to leave some vertical whitespace above
\def\Defmacro #1 #2 {\Dodocf {#1} {#2} {Macro}}
\def\Defun #1 #2 {\Dodocf {#1} {#2} {Function}}
\def\Defgeneric #1 #2 {\Dodocf {#1} {#2} {Generic Function}}
\def\Defmethod #1 #2 {\Dodocf {#1} {#2} {Primary Method}}
\def\Defbeforemethod #1 #2 {\Dodocf {#1} {#2} {:Before Method}}
\def\Defaftermethod #1 #2 {\Dodocf {#1} {#2} {:After Method}}
\def\Defaroundmethod #1 #2 {\Dodocf {#1} {#2} {:Around Method}}
\def\Defvar #1 {\Dodoc {#1} {Variable}}
\def\Defvarv #1 #2 {\Dodocv {#1} {#2} {Variable}}
\def\Defconst #1 {\Dodoc {#1} {Constant}}
\def\Defconstv #1 #2 {\Dodocv {#1} {#2} {Constant}}
\def\Defclass #1 {\Dodoc {#1} {Class}}
\def\Defclassv #1 #2 {\Dodocf {#1} {#2} {Class}}
\def\Defslot #1 {\Dodoc {#1} {Slot}}
\def\Defslotv #1 #2 {\Dodocv {#1} {#2} {Slot}}
\def\Definitarg #1 {\Dodoc {#1} {Initarg}}
\def\Deftype #1 {\Dodoc {#1} {Type}}
\def\Defoption #1 {\Dodoc {#1} {Option}}
\def\Defcondition #1 {\Dodoc {#1} {Condition}}
\def\Defrestart #1 {\Dodoc {#1} {Restart}}

%% Use these when you don't want to leave any vertical whitespace above
\def\dodoc#1#2{\Vskip .1pt!%
	\outdent{$\Rightarrow$}\index{{\tt #1}!#2|ital}\index{#2!{\tt #1}|ital}%
	{\tt #1\hfill\hbox{\brac{\it #2\/}}%
		\Vskip .1pt!}}
\def\dodocf#1#2#3{\Vskip .1pt!%
	\outdent{$\Rightarrow$}\index{{\tt #1}!#3|ital}\index{#3!{\tt #1}|ital}%
	{\hangindent=1cm\hangafter=1%
		\tt #1\hfill\hbox{\brac{\it #3\/}}\linebreak%
		{\raggedright{\sl #2}\hfill\Vskip .1pt!}}}
%\def\dodocv#1#2#3{\vbox spread -10pt {\Vskip .1pt!%
\def\dodocv#1#2#3{\vbox spread -2pt {\Vskip .1pt!%
	\outdent{$\Rightarrow$}\index{{\tt #1}!#3|ital}\index{#3!{\tt #1}|ital}%
	{\hangindent=1cm\hangafter=1%
		\tt #1\hfill\hbox{\brac{\it #3\/}}\linebreak%
		{\raggedright{\tt #2}\hfill\Vskip .1pt!}}\vss}}

%% Use these when you don't want to leave any vertical whitespace above
\def\defmacro #1 #2 {\dodocf {#1} {#2} {Macro}}
\def\defun #1 #2 {\dodocf {#1} {#2} {Function}}
\def\defgeneric #1 #2 {\dodocf {#1} {#2} {Generic Function}}
\def\defmethod #1 #2 {\dodocf {#1} {#2} {Primary Method}}
\def\defbeforemethod #1 #2 {\dodocf {#1} {#2} {:Before Method}}
\def\defaftermethod #1 #2 {\dodocf {#1} {#2} {:After Method}}
\def\defaroundmethod #1 #2 {\dodocf {#1} {#2} {:Around Method}}
\def\defvar #1 {\dodoc {#1} {Variable}}
\def\defvarv #1 #2 {\dodocv {#1} {#2} {Variable}}
\def\defconst #1 {\dodoc {#1} {Constant}}
\def\defconstv #1 #2 {\dodocv {#1} {#2} {Constant}}
\def\defclass #1 {\dodoc {#1} {Class}}
\def\defclassv #1 {\dodocf {#1} {Class}}
\def\defslot #1 {\dodoc {#1} {Slot}}
\def\defslotv #1 #2 {\dodocv {#1} {#2} {Slot}}
\def\defaccessor #1 {\dodoc {#1} {Accessor}}
\def\definitarg #1 {\dodoc {#1} {Initarg}}
\def\defoption #1 {\dodoc {#1} {Option}}

%% for creating examples:   (no indexing)
\def\egdoc#1#2{\Vskip .1pt!%
	\outdent{$\Rightarrow$}%
	{\tt #1\hfill\hbox{\brac{\it #2\/}}%
		\Vskip .1pt!}}
\def\egdocf#1#2#3{\Vskip .1pt!%
	\outdent{$\Rightarrow$}%
	{\hangindent=1cm\hangafter=1%
		\tt #1\hfill\hbox{\brac{\it #3\/}}\linebreak%
		{\raggedright{\sl #2}\hfill\Vskip .1pt!}}}
\def\Egdocf#1#2#3{\Vskip \bigskipamount!%
	\outdent{$\Rightarrow$}%
	{\hangindent=1cm\hangafter=1%
		\tt #1\hfill\hbox{\brac{\it #3\/}}\linebreak%
		{\raggedright{\sl #2}\hfill\Vskip .1pt!}}}
\def\egdocv#1#2#3{\Vskip .1pt!%
	\outdent{$\Rightarrow$}%
	{\hangindent=1cm\hangafter=1%
		\tt #1\hfill\hbox{\brac{\it #3\/}}\linebreak%
		{\raggedright{\tt #2}\hfill\Vskip .1pt!}}}

\def\egvar #1 {\egdoc {#1} {Variable}}
\def\eggeneric #1 #2 {\egdocf {#1} {#2} {Generic Function}}
\def\Eggeneric #1 #2 {\Egdocf {#1} {#2} {Generic Function}}
\def\egmethod #1 #2 {\egdocf {#1} {#2} {Primary Method}}
\def\egslotv #1 #2 {\egdocv {#1} {#2} {Slot}}
\def\egaccessor #1 {\egdoc {#1} {Accessor}}
\def\eginitarg #1 {\egdoc {#1} {Initarg}}

\def\optional{{\tt\&optional}\ }
\def\rest{{\tt\&rest}\ }
\def\key{{\tt\&key}\ }
\def\allow{{\tt\&allow-other-keys}\ }
\def\body{{\tt\&body}\ }

\def\gap{\bigskip}

%% Common words and phrases
\def\geco{\mbox{{\sc geco}}}
\def\Geco{\mbox{{\sc Geco}}}
\def\ie{{\it i.e.}}
\def\Ie{{\it I.e.}}
\def\eg{{\it e.g.}}
\def\Eg{{\it E.g.}}
\def\etc.{{\it etc.}}

	\makeindex

% fancyhdr (was fancyheadings) stuff
\usepackage{fancyhdr}
\pagestyle{fancy}   %%%% \pagestyle{fancy}	%for fancyheadings.sty.  Must be after any changes to \textwidth
\renewcommand{\sectionmark}[1]{\markboth{\thesection\ #1}{\thesection\ #1}}
\renewcommand{\subsectionmark}[1]{\markboth{\thesubsection\ #1}{\thesubsection\ #1}}
\fancyhead[EOL]{\bf GECO \gecoversion}	%%%% \lhead{\bf GECO \gecoversion}
\fancyhead[EOR]{\bf\thepage}					%%%% \rhead{\bf\thepage}
\fancyfoot[EOL]{\today}								%%%% \lfoot{\today}
\fancyfoot[EOR]{\bf\leftmark}					%%%% \rfoot{\bf\leftmark}
\fancyfoot[EOC]{}									%%%% \cfoot{}  %defaults to page number
\renewcommand{\headrulewidth}{0.4pt}
\renewcommand{\footrulewidth}{0.4pt}

