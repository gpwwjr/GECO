\chapter{Introduction}

\section{Purpose}

Genetic algorithms (GAs) \cite{ga:holland92,ga:goldberg} cover a broad
category of robust adaptive techniques. They have been used to solve an
increasingly large variety of difficult problems, including parameter
optimization, combinatorial optimization, rule learning, pattern recognition,
automatic programming, adaptive control, \etc.

The GA research community has produced a number of reusable GA implementations, \eg, 
\cite{ga:genesis,ga:genesis-ug,ga:sga,ga:genitor:Whitley-Kauth-88,ga:gal} and many more,
but many of them implement a specific approach to GAs which their authors
prefer, or which reflects their research orientations. Though these
implementations can be adapted to solving different problems, the amount
of effort required to customize them generally increases dramatically as
the proposed changes vary from the original visions of their
implementors.

Developing a GA-based application often requires a great deal of 
experimenting --- trying variations on the genetic operators, the selection 
algorithm, establishing various parameter settings, \etc.  Regardless of 
what language is used for implementing the final system, it often makes a 
great deal of sense to develop a prototype first, establishing the best 
kind of GA for the application and gaining a good understanding of the 
effects of varying each of the parameters, before proceeding to create 
the production implementation.

\Geco\index{GECO} (Genetic Evolution through Combination of Objects) is an
attempt to construct a more flexible framework for prototyping GAs. It makes
extensive use of object-oriented implementation techniques to
provide this flexibility, and it was designed from the outset with the goal of
being easily customized and extended.

Although \geco\ is intended for developing more-or-less classical GAs, it 
appears to be sufficiently flexible to be used for prototyping some related 
classes of algorithms: Learning Classifier Systems (LCSs) 
\cite{gbml:holland-reitman,gbml:holland-induction}, Genetic Programming 
(GP) \cite{gp:koza}, Evolution Strategies (ESs) \cite{es:survey}, and 
Evolutionary Programming (EP) \cite{ep:ds+go=ep}.

\section{State of the System}

\Geco\ was initially developed in the early 1990s, and used in some of the author's
research around that time. But since then it has been mostly ignored
until just recently. The author is now retired, and
has decided to dust it off and update it for compatability with contemporary lisp
development tools.
It is still does not provide
all the features required for it to be considered a complete GA toolkit. But
where it lacks functionality, the author has attempted to make its structure sufficiently 
robust and flexible that it can be easily extended to implement new
capabilities. Much like other evolutionary
processes, I hope to improve \geco\ as I learn through feedback from its
community of users.

\Geco\ is copyrighted free software. The complete source code is available
directly from the author, and I will endeavor to make it available from online archives on
the Internet.\index{\geco!availability}

This version of the \geco\ documentation refers to version
\gecoversion\index{\geco!version} of the software.

\section{Why Lisp?}

Since GAs are computationally expensive, most implementations are in
`conventional' programming languages such as {\tt C} ``for efficiency.'' Lisp is
a much more convenient language for prototyping, since almost all
implementations are highly interactive and provide a great deal of support for
the development process. Common Lisp has become the standard industrial
strength dialect in the United States, and it has many high quality
implementations on a wide variety of platforms. Furthermore, the compilers are
capable of producing code which often rivals `conventional' language
implementations for speed and efficiency. Common Lisp is also one of the most
portable languages available today --- in many ways more portable than {\tt C}
or Ada. The addition of the Common Lisp Object System
(\concept{CLOS}) to the Common Lisp language \cite{cl:steele2,cl:keene} has
provided a full-featured object-oriented extension to the language, which
greatly enhances its capability for prototyping and writing extensible
libraries.


\section {Notational Conventions}

This document follows several notational conventions for the sake of
conciseness, and to assist the reader in understanding usage in context.

\subsection {Terms and Concepts}

Most new (and some common) terms and concepts which are used in this document
are set in a {\sf sans serif} font when they are introduced and/or defined, or when
it seems important to emphasize that the word or phrase has a specific meaning
in context.

\subsection {Source Code}

All source code is set in a {\tt typewriter} style font, whether it
appears in the body of the text, or set off from the body of the text as
an example.  This includes names of Common Lisp or \geco-defined
entities, such as names of functions, classes, \etc., which will also be
set in {\tt typewriter} style font.

\subsection {File Names, Network Paths, \ldots}

All such references will be set in a {\tt typewriter} style font.

\subsection {Descriptions of \Geco-defined Entities}

The syntactic descriptions of functions, methods, variables, classes, and other
definitions are presented in a distinctive format, similar to that used in Guy
Steele's \ital{Common Lisp: The Language} \cite{cl:steele2}, and many other
similar documents since. This stylized description is generally followed by
narrative body text explaining the intended usage and semantics. References in
the body text to components of the syntactic description will appear in the same
type style in which the component appears in the syntactic description.

The first line of one of these syntactic descriptions is always signalled by a
$\Rightarrow$ symbol appearing in the left margin; this line is referred to
below as a \concept{flag line}, since the symbol in the margin helps to
{\em flag} the readers attention. This flag line specifies the name of the
definition against the left margin, with the type of definition in italics and
brackets against the right margin. For example:

\egvar {example-variable}

For functions, generic functions, and macros, any arguments appear on indented
lines immediately following the flag line, with the argument names set in
\slant{slanted} type. Common Lisp lambda-list keywords (\eg, 
\verb_&optional_, \verb_&key_) are set in {\tt typewriter} style type,
since they are literal language constructs in the definition (though they
do not appear when the form is used). In addition, any \verb_&optional_,
\verb_&key_, or \verb_&rest_ arguments which have default values are shown
parenthesized, with the default value specification shown in {\tt
typewriter} style type, since it is a Common Lisp literal.

\eggeneric {example-function} {(arg1 arg2 \optional arg3)}

For functions and methods, the second and subsequent lines specify the argument list, but also
shows the specialization of arguments. \Ie, specialized arguments are shown in
parentheses, with the specializing class following the argument, and in {\tt
typewriter} style font, since it is a literal reference to a class.

\egmethod {example-function} {((arg1 \cl{class1}) arg2
                               \optional (arg3 \cl{'default-value}))}

Variables, constants, and slots are specified in a similar manner. However since
these entities may have default or initial values, such values are shown on
indented lines immediately following the flag line. Similarly, class description
flag lines may be followed by indented lines identifying any superclasses of the
class.

\egslotv {example-slot} {default-slot-value}

Slot descriptions always follow the description of the class on which they are
directly defined (descriptions of inherited slots are not repeated).
Accessor functions and initargs for slots (if any) are specified on flag
lines immediately following the descriptions of their slots.

\egaccessor {example-slot-accessor}
\eginitarg {:example-initarg}

\subsection {The Index}

All of \geco's definitions are doubly indexed, by both name and type of
definition (\eg, function, class, slot, \etc.). In addition, all references to
\geco-defined entities in the body of the text are indexed.  To aid the
user in finding the definition in a potential forest of references, the
page number of the definition(s) appears in \ital{italics}. 
Note that in some cases, there may be more than one definition of a
method, and the generic function definition is generally repeated in
these cases as well.


\section {Acknowledgments}

I want to thank Randy Fennel and Al Underbrink for their help in beta-testing
the software, and for their comments on early drafts of this documentation. Of
course I am solely responsible for any errors which surely remain. I also want
to thank John Koza for kindly permitting the reproduction of his implementation
of the Park-Miller randomizer, Kate Juliff for inspiring the multi-chromosome
features, and Larry Yaeger for publishing the \cl{C} implementation of gray code
translation after which my implementation
(section~\ref{section:gray-code-translation}) is patterned. 
